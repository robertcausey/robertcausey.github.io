\documentclass[11pt]{article}
\usepackage{geometry}
\usepackage[comma,sort&compress]{natbib}
\usepackage{rotating}
\usepackage{graphics}
\usepackage{amsmath}
\usepackage{epsfig}
\usepackage{enumitem}
%\usepackage{picins,graphicx}
\usepackage{txfonts}
\usepackage[english]{babel}
\geometry{letterpaper, margin=1 in}
\renewcommand{\rmdefault}{ptm} % Arial
\renewcommand{\sfdefault}{phv} % Arial
\def\changemargin#1#2{\list{}{\rightmargin#2\leftmargin#1}\item[]}
\let\endchangemargin=\endlist 
\begin{document}
\sloppy \rmfamily
\begin{center}\textbf{School of Food and Agriculture\\~\\2020-2021 Annual Report\\~\\(June 30, 2021. Prepared by Robert Causey)}\end{center}
\textbf{Executive Summary}\\~\\
Pandemic\\-brightspace, -online education, -hands-on at witter center
Define tomorrow\\-food systems major
Nutrition Search\\-appears successful
Two faculty early tenure\\-Juan and Jen
Advising center\\-relocation tom roger clapp
Students recruitment and retention\\-report numbers
%The year 2020 - 2021 saw the School of Food and Agriculture successfully adapt to the Covid-19 Pandemic. Our faculty successfully migrated their in person courses to Brightspace or other online formats. Meanwhile we were able to continue offer core in person teaching functions, such as hands-on activities at the Witter Center, esssential research at farms, and aspects of the Dietetic Internship. Two of our faculty (JUAN ROMERO and JEN PERRY) successfully applied for early tenure. This speaks to the nurturing environment we have provided young faculty. SUSAN SULLIVAN led an initiative with JEN PERRY and RACHEL SCHATTMAN to propose a Food System's major that will bridge all the three academic disciplines of the three of the former disciplines within the unit (FSN, PSE, AVS). We responded to the President's Define Tomorrow call with a proposal, led by ERIC GALLANDT, to address resiliency of Maine's food system in the context of Covid-19 and climate change. We won approval to hire a Nutritional Biochemist, the search for which is now its final stages with two well qualified candidates in the final selection. This successful recruiting effort speaks directly to the third goal of the President's Strategic Vision and Values framework, that ``The university will be a rewarding place to live, learn and work"  and to the R \& D goal for FY2020 - FY2024 ``To make Maine the best state in the nation in which to live, work, and learn by 2030". Despite the necessary upheaval, our faculty have made way for the new Advising Center which is to be based in space formerly occupied by our faculty in Deering Hall. Meanwhile, we were able to consolidate Sustainable Agriculture Faculty in the top floor of the Roger Clapp Greenhouses. Despite the pandemic our overal recruitment efforts appear to have achieved growth in the Environmental Horticulture and Animal and Veterinary Sciences majors. A reduction on FSN majors, reflecting a national trend, is partially offset by growth in our UMaine Gold online Masters Certificate programs in Food Science and Nutrition, led by MONA THERRIEN.  Our achievements and challenges are described in more detail below using the Strategic Visions and Values framework. 
\begin{description}[topsep=0pt, noitemsep]
\item[\textbf {1.}] \textbf{We will support and grow Maine’s economy through new discoveries and by building a workforce whose members are engaged in their communities and prepared for lifelong success.}
%\begin{description}[label=\textbf{\alph*.}] 
\begin{description}[topsep=11pt, noitemsep]
\item[\textbf{1.1}] \textbf{We will welcome and support all learners and engage them in experiential learning.}\\~\\
	TIM BOWDEN - STUDENTS IN LAB AQUACULTURE, NEW CLASS AVS 411/511 ADVANCED AQUACULTURE 
	STEPHANIE BURNET - ADAPTED GREEN HOUSE TO COVIDi- PSE 109 online, FFA, earlycollege
MARY ELLEN CAMIRE extensive teaching umaine gold, extensive graduate program, 	
SUE ERICH Soils teaching - importance of it.
ERIC GALLANDT sustainable ag teaching, UMaine Greens, organic farming
CHARLENE GRAY - ecological landscape design teaching using native plants and efftive drainage. 
JAY HAO -Graduate Education
SUE ISHAQ - New classes - Introduction to Animal Microbiomes and DNA sequencing laboratory while carrying Capstone.  Advisor for 10  senior topics, co-chair for 5 and on committee for 1. 
PAULINE KAMATH Graduate and unbdergraduate course in wildlife, large class in Lab animal companion medicine, great success in promoting undergraduates in research 
DOROTHY KLIMIS-ZACAS - Classes in lipid metabolism, heart disease, and nutritional biochemistry and a strong graduate program
ANNE LICHTENWALNER - Experiential learning and research in veterinary diagnostics. Capstone, Honors,and CUGR.	 
JADE MCNAMARA Classes in community nutrition and behavior. I supervised a Maine Top Scholar, Ashley Reynolds. She completed a project that explores intuitive eating behavior in college students and had her abstract
EILEEN MOLLOY - human nutrition and the communication of food and nutrition concepts for improved health Food preparation classes nationally accredited Didactic Program in Dietetics. Navigate to set up advising appointments for all continuing students. I use the commerical kitchen in Hitchner Hall and work with the Pilot Plant manager to make sure it is in good working condition. 
BRIAN PERKINS Teaches Brewing with Food Science.
JENNIFER PERRY Classes in Microbiology,  food safety, sanitation.
BRYAN PETERSON propagation, evaluation, ecophysiology, and population genetics of woody plants for horticulture. 
JUAN ROMERO Classes Animal nutrition     Forage Science and Range management Ruminant nutrition
RACHEL SCHATTMAN coursework in sustainable food systems
SUSAN SULLIVAN Together with a committee in the School of Food and Ag, prepared and submitted a Program Request to the Provost for a multicampus, multidisciplinary BS degree in Food Systems. Didactic program in dietetics.
MONA THERRIEN - Dietetic internship,expertise in diabetes, nutrition support, chronic kidney disease, and nutrition for the aging population. UMaine Gold Coordinator
DENISE SKONBERG- Classes in food product development and utilization of aquatic food resources.
JIM WEBER Classes in Reproductive physiology of domestic animals, including horses and ruminants.
%Under the direction of Associate Director SUE SULLIVAN, SFA's undergraduate coordinators, EILEEN MOLLOY, CHARLENE GRAY and JIM WEBER proactively reviewed our undergraduate curricula in Food Science and Human Nutrition (FSN), Environmental Horticulture and Sustainable Agriculture (ENH/SAG), and Animal and Veterinary Sciences (AVS). Recent innovations include development of modules for the SAG curriculum, led by ERIC GALLANT, SUE ERICH, and RACHEL SCHATTMAN, and modification of the AVS curriculum to include new classes of SUE ISHAQ and an equine concentration within the undergraduate Animal Science curriculum \textbf{(SVV 1.1.1 \& 1.1.2)}. All our undergraduate curricula are focused on experiential learning, with most students performing research for their capstone experience, as employees, or as student investigators in the Center for Undergraduate Research or the Honors College. As part of the modularization of courses in SAG, we are increasing opportunities for microcredentials, along with the new UMaine GOLD program in Human Nutrition and Food Technology, and the Didactic Program in Dietetics (EILEEN MOLLOY coordinator) and the Dietetic Internship (MONA THERRIEN coordinator; \textbf{SVV 1.1.3, 1.1.4, \& 1.1.6)}. Other faculty in FSN (MARY ELLEN CAMIRE, DOROTHY KLIMIS ZACAS, and DENISE SKONBERG) contribute to the GOLD program, which has been coordinated by KATHRYN YERXA.  BRIAN PERKINS successfully compiled and filed an application for the Institute of Food Technologists approval of the Undergraduate program in Food Science. Enrollment in our AVS and FSN undergraduate program remains high, with high rates of acceptance, and scholarships for students in financial need \textbf{(SVV 1.1.5)}.\\~\\      

\item[\textbf{1.2}] \textbf{We will create new knowledge and apply innovative research and scholarship to enrich lives.}\\~\\
TIM BOWDEN ACQUACULTURE
STEPHANIE BURNETT -aeroponics, at home sensors
LILY CALDERWOOD - blueberry research weeding, seacrop16, weather, climate change gall midge 100k in grants, articles scientific americvan and associated press 
MARY ELLEN CAMIRE - 200k extramural grants
ROBERT CAUSEY - horse program students numbers and MYSQL records development
SUE ERICH Preservation abd stabilization of organic matter in soil
ERIC GALLANDT - Weed ecology and control through organic means. PI pn 3 continuing USDA grants 
		JAY HAO Research to sustain the potato industry - potato diseases. Blackleg and soft rot pathogenesis, genetic resistance. Funding through the Maine Potato board, USDA and private sources  
SUE ISHAQ - CO-PI on 3 continuing grants, PI on \$1.5 million extramural 
PAULINE KAMATH - PI of 3 continuing federal grants, and co-pi of additional 3 continuing federal grants. Received \$800k of federal grants this year.
DOROTHY KLIMIS-ZACAS Almost \$300k in requested funding, prestigious book by Royal Society of Chemistry.
ANNE LICHTENWALNER - Funding from MTAF and NSF - training grant 
JADE MCNAMARA relationship between nutrition and food literacy and dietary behaviors \$177,676 USDA - Utilizing Community- Based Participatory Research to Increase Health Related Quality of Life in College Students

RENAE MORAN Tree fruit production and physiology Enhancing peach production through better variety selection, planting in low-risk sites and understanding cold hardiness.  Reducing postharvest fruit losses of Honeycrisp.  pruning techniques to increase consistency in yield of honeycrisp apples USDA and MDA current projects, Awarded \$55K for Peach Variety Testing and Development for a Local Market 
TSUTOMU OHNO The investigation of how molecular structure of soil organic matter interacts with soil processes such as phosphorus bio-availability and carbon stabilization in natural and managed ecosystems. ultrahigh-resolution mass spectrometry chemical force spectroscopy  atomic force microscopy computational chemistry
BRIAN PERKINS Food Analytical Chemistry, specializing in the development of chromatographic techniques (LC & GC). Detection of bio-active chemical metabolites in chaga and fermented products (kombucha, beer, fermented vegetables).
JENNIFER PERRY Research in Microbiology, food product development, fermentation, food safety, sanitation. Invasive green crab and wild blueberries.PI on 7 current projects ( 3 federal 3 stakeholder) and co pi on 4 federal projects Received \$40K federal grant Class Market development as mitigation strategy for ecosystem damage and predation by invasive green craband a co-pi on almost \$1 million submitted in ther last year. 
BRYAN PETERSON serviceberries (Amelanchier spp.), edibe honeysuckle (Lonicera villosa) and other woody taxa that are indigenous to Maine, absent from the horticulture industry, and have potential for horticultural use. Investigation of invasive potential of nonnative honeysuckle (Lonicera caerulea) recently introduced to North America for berry production, and comparison to the native Lonicera villosa for horticultural attributes. Design and evaluation of a novel propagation systems for Maine's horticulture industry.  Continuing support Submist to Propagate Nursery Crops by Stem Cuttings Horticultural Research Institute

GREG PORTER creates new potato varieties using conventional plant breeding
techniques at our facilities in Orono and Presque Isle. Our goal is to develop new potato varieties that provide improved quality and marketing opportunities for
potato growers, as well as to help solve pest management problems.

Caribou Russet’s cash farm value to ME
seed growers was ~\$3.9 M during 2020 and the estimated cash farm value when this seed crop is planted, grown and harvested in 2021 is ~\$33.1M. It is also being
evaluated and adopted in many other countries around the world. Hamlin Russet (tested as AF4124-7) was released by ME during 2020 for early fry processing and
russet fresh market. It has moderate scab resistance. Certified seed acreage rose to 83 acres (#78th in the US) during 2020.

PI on 9 continuing projects. Over \$600K awarded this year. My research has shown that soil amendments and crop rotation can be used to build soil organic matter and supply nutrients to maintain high value potato crops.

JUAN ROMERO Dr. Romero aims to expand the understanding of the factors that affect forage quality and conservation in order to develop novel additives that will improve profitability of livestock producers. Specifically, his program focuses on the methodological use of fungal enzymes to solve specific issues in silage production and the development of biologically-based additives to enhance the stability of conserved forages.
3 continuing grants, 2 MFAC, one USDA

Funded this year USDA \$160K A multiregional approach to balancing milk and forage quality tradeoffs in organic dairies feeding highlegume diets

RACHEL SCHATTMAN agricultural resilience in a changing climate while simultaneously protecting natural resources. In pursuit of this goal, I work with specialty crop producers and agricultural advisors to identify and address production challenges, specifically through the lens of climate change adaptation. Received almost \$240,000 for the following 3 projects:Climate change adaptation and mitigation research, outreach, and education for land managers in Maine and the Northeast; Enhancing adoption of regenerative agriculture practices in U.S. wheat farming system; The impact of the local food system and natural environment on rural food security and health outcomes during the COVID-19 pandemic

DENISE SKONBERG Specific focus areas include utilization of invasive green crab; seaweed quality assessment and product development; high pressure processing; characterization of functional properties of previously cooked crustacean meat; processing of high-value molluscan seafood.

JIM WEBER Mechanism of soft tissue and bone healing in mammalian models in response to additively manufactured metallic medical implant materials.
Management of sheep flocks in northern climates for decreased parasite-related losses.
%Research conducted by faculty in SFA is driven by the land grant mission, and has direct impact on the lives of Maine citizens, the country and the world \textbf{(SVV 1.2.1)}. Consistent with the President's 2019 Research and Development Plan to ``Make Maine the best state in the nation in which to live, work, and learn by 2030" we address basic and applied science that changes the situation on the ground in Maine. At the molecular level, STOM OHNO continues to address the attachment of organic matter to inorganic matrices in soil, to increase our understanding of carbon trapping in the soil and its potential to mitigate climate change. SUE ERICH examines soil quality and the stabilization of soil organic matter, which helps to reduce fluxes of carbon dioxide into the atmosphere, potentially moderating rates of climate change. ERIC GALLANDT explores methods of weed control which are ecologically sound, employing a variety of mechanical methods that impose stresses at multiple points in a weed’s life cycle and maintain soil quality. Our newest faculty member, RACHEL SCHATTMAN, studies the effects of water usage and agricultural irrigation irrigation through the lens of adaptation to climate change. \\~\\
%Our faculty in Animal and Veterinary Sciences also study the underlying mechanisms affecting human and animal health \textbf{(SVV 1.2.1)}. PAULINE KAMATH's research explores our understanding of host-pathogen co-evolution and infectious disease transmission dynamics, primarily in parasitic diseases affecting human, wildlife, and domestic animal health. JIM WEBER explores the seasonality of the parasite life-cycle in sheep, as a way to prevent economic loss for sheep producers, and ANNE LICHTENWALNER has evaluated several alternative antimicrobials against a common pathogen of sheep that causes caseous lymphadenitis (CL). In addition she collaborates with Maine Inland Fisheries and Wildlife (IFW) to evaluate moose and deer parasites and pathogens. SUE ISHAQ, who was hired in 2019, similarly explores the gut microbiome of wild and domestic animals, and the critical role of microbial communities in animal and human health and disease.\\~\\  
%Faculty in Human Nutrition continue to perform basic research at both the molecular, and behavioral level \textbf{(SVV 1.2.1)}. DOROTHY KLIMIS-ZACAS has shown that anthocyanin and phenolic acid extracts from wild blueberries reduce vascular damage that underlies coronary heart disease and wound healing (diabetes, burns, etc). This potentially has great economic impact for the Blueberry industry. Also, JADE MCNAMARA  developed a 120 item nutrition literacy survey for college students and an environmental assessment for low-income, rural communities. This work is necessary to address food security in Maine, an issue which has become critical during the Covid-19 pandemic.\\~\\
%Applied science by SFA faculty has created significant economic opportunity for Maine's agricultural and food sectors \textbf{(SVV 1.2.2)}. The development of the Caribou Russet and Pinto Gold potato varieties by GREG PORTER has created a commodity with an economic value of \$9.1 million for Maine growers. In addition, JAY HAO has successfully lessened the impacts of potato diseases, in part through evaluating different chemical compounds and varieties to obtain good candidate of chemical products or potato germ plasm for disease resistance breeding. In part, through his work, the pathogen distribution and evolution of blackleg and soft rot during the outbreak in Northeastern US was elucidated.\\~\\
%JENNIFER PERRY has developed methods to improve the microbiologic safety of a large variety of Maine food products, including fruit, vegetables, seafood and dairy products. DENISE SKONBERG has also explored the development of seaweed processing and the use of high pressure treatment to kill pathogens in seafood. MARY ELLEN CAMIRE has worked to develop healthful foods for older adults, and to promote foods produced in Maine. She has also explored the health benefits of dietary fiber and anthocyanins and explored consumer attitudes towards healthy foods, contributing to development of new food products. Her work shows that Consumers are unsure of how to incorporate seaweed into their diets, but have gained confidence in preparing fermented foods safely at home. \\~\\
%In AVS, JUAN ROMERO is working to patent lignosulfonates as hay preservatives. The current hay preservative market in the US is ~\$350 million for a ~\$16 billion annual hay market in the US alone \textbf{(SVV 1.2.2)}. TIM BOWDEN has assessed the oyster population in Damariscotta River Estuary, showing gene flow between commercial and natural bed populations over time. He has also studied the prevalence of the oyster parasite `` MSX" in Damariscotta River Estuary. His work shows that MSX prevalence has been greatly reduced, indicating that the changes in management practice in the Damariscotta Estuary may have aided in the removal of this important oyster pathogen from the system. \\~\\
%All of the above research faculty have been involved with undergraduate and graduate education, contributing to workforce development for Maine's economy in entry level, technical and leadership positions \textbf{(SVV 1.2.3 \& 1.2.4)}.          
\end{description}
\item[\textbf{2.}] \textbf{We will continue to provide accessible and affordable education, research and service through processes that ensure effectiveness, efficiency and quality.}
\begin{description}[topsep=11pt, noitemsep]
\item[\textbf{2.1}] \textbf{We will grow and advance partnerships to catalyze the cultural, economic and civic future of Maine and beyond.}\\~\\
	Stephanie Burnett editorships, reviews, memberships
	LILY CALDERWOOD - bluewave solar, wild blueberry commission, service
	MARY ELLEN CAMIRE - Keynote address American Heart Association, Aging ME and collaborations with Bowling Green State University, UColo, and Maine Business school.Professional memberships.
	SUE ISHAQ - organized the Microbes and Social Equity working group,leading the development of a journal special collection, and was guest editor on  approved by mSystems in 2021 
	ANNE LICHTENWALNER Maine Vector-borne Disease Working Group, the Food Safety Working Group, the Poultry Health Advisory Committee and the Rabies Committee.
		JADE MCNAMARA Led a break-out session on how to use critical thinking in higher education nutrition course at the Society of Nutrition Education and Behavior annual meeting, with an attendance of 500+ nutrition professionals.
EILEEN MOLLOY Academy of Nutrition and Dietetics
Maine Academy of Nutrition and Dietetics

RENAE MORAN Collaborations with UNH, Cornell, WSU
TSUTOMU OHNO Collaborations with Penn and Princeton
JENNIFER PERRY Collaborations with University of New England 
GREG PORTER My research on potato cropping systems, supplemental irrigation, and nutrient management has resulted in improved recommendations for potato producers and processors.  Improved management practices and varieties help potato growers produce an affordable, high quality potato crop with as little environmental impact as possible.  More profitable agricultural systems can help stimulate the rural economy and maintain open space. My research helps growers use inputs efficiently thus minimizing impact on the environment and conserving soil resources.

JUAN ROMERO Relationship with Lallemand, Sappi, BASF, U Delaware, Florida, NC State, UNH, Virginia Tech

RACHEL SCHATTMAN Results from the 2020 Maine Agriculture Drought Survey: I led a survey in the winter of 2020-2021, targeting Maine’s biggest commodities (potatoes, wild blueberries) and other important agricultural sectors (diversified vegetable and small fruit, tree fruit, maple, hemp, livestock and dairy, nursery and greenhouse producers). The survey instrument was collaboratively developed by a team of agricultural advisors and service providers from the Maine Department of Agriculture, Conservation, and Forestry (MDACF), the University of Maine Extension, the University of Maine School of Food and Agriculture, and the Maine Organic Farmers and Gardeners Association (MOFGA)
My research on drought impacts in the state of Maine has been used by the Maine Water Board to assess the impact of the 2020 drought on agriculture. The Board
has been discussing advocating for funding to support farmers' ability to access water during periods of drought without degrading natural resources or ecosystems.  The report offers key recommendations for how to accomplish this goal.  More broadly, my work seeks to accelerate the rate of climate adaptation and mitigation in land use sectors. I predict that, without support or resources, many land managers will be ill prepared to contend with climate change challenges. By investing in research that (a) documents limitations to adaptation and mitigation, (b) supports land manager communities of practice around climate change adaptation and mitigation, and (c) integrates the best climate science with the place-based expertise of farmers and foresters, we can better prepare for the challenges ahead. By extension, this will support the long term viability of agriculture and forestry
MONA THERRIEN Attended Food \& Nutrition Conference and Expo's FNCE virtual Dietetic Internship Fair fall of 2020. This national conference allows Dietetic internship programs to recruit from over 2000 prospective students from all over the country
%Faculty in SFA have a variety of partnerships, including research consortia, industry stakeholders, and other educational institutions in Maine. Maine's potato industry working closely with GREG PORTER, has greatly increased the value of it's holdings through development of the Caribou Russet and Pinto Gold \textbf{(SVV 2.1.1)}. In part, through our relationship with Cooperative Extension and 4H, faculty such as KATE YERXA, DAVE MARCINKOWSKI, ANNE LICHTENWALNER, LILY CALDERWOOD, and HOBSON MACHADO, have been closely involved with local communities, businesses, and schools to address need in community health and nutrition, dairy farm and agricultural profitability, zoonotic diseases, and educational programming \textbf{(SVV 2.1.2)}. More recently we have proposed, with the advent of Covid-19, UM system-wide cooperation with other campuses as part of our Define Tomorrow pre-proposal ``Building a resilient food system for Maine" \textbf{(SVV 2.1.3)}.\\~\\      \newpage 
\item[\textbf{2.2}] \textbf{We will optimize management of our infrastructure and enhance it to support the realization of our vision.}\\~\\
	STEPHANIE BURNETT chair or green house committee
	MARY ELLEN CAMIRE maintenance and reorganization of sensory testing laboratory.
	SUE ERICH Analytical Laboratory
	ROBERT CAUSEY Maintenance of Equine Herd and Chute Center
	CHARLENE GRAY Landscape Horticulture Studio
	SUE ISHAQ - Manages lab for molecular/microbiology, cell culture, and DNA sequence analysis
	DOROTHY KLIMIS-ZACAS Angiogenesis laboratory
	ANNE LICHTENWALNER Director, Maine Veterinary Diagnostic Laboratory
	RENAE MORAN - Highmoor Farm
BRIAN PERKINS My research laboratory is a large (1200 square foot) space equipped with highly sought analytical instrumentation. I am currently working closely with twelve graduate and undergraduate students from our school (SFA) and other UMaine departments/schools. Current projects include analysis of biogenic amines in a number of novel fermented foods, method development for analysis of bioactive compounds in blueberries and purification and identification of a naturallyoccurring plant compound to control ciliates in the seaweed industry. Training/supervising these students and maintaining these instruments is a time consuming
activity.
BRYAN PETERSON - Greenhouse
JUAN ROMERO - Fistulated cow
RACHELL SCHATTMAN I supervise the University of Maine Agroecology Lab, which is located in Deering Hall. I also manage research field plots at Rogers Farm, and conduct research at greenhouses located at the UMaine Extension Plant Diagnostic Lab.  adaptation.
%The School of Food and Agriculture either controls, or uses heavily, certain infrastructure assets which align with the University's mission. The Analytical Laboratory, supervised by SUE ERICH, plays an important role in sustaining the agricultural economy throughout the North East, and employs several highly skilled staff in the analysis of submitted soil samples on a fee for service basis in a relatively low foot print in Deering hall. In addition, STEPHANIE BURNETT and BRYAN PETERSON are heavily involved with the use and  maintenance of the Roger Clapp Green Houses, which sustain a valuable ornamental plant collection, and make possible nationally recognized horticultural research and educational programming for UMaine students and the community. The recently completed Veterinary Diagnostic laboratory, supervised by ANNE LICHTENWALNER, provides state of the art diagnostics on a fee for service basis. In addition, the consolidation of several species of domestic animals at the Witter Center sustains our Animal Science and Pre-Veterinary programs led by DAVE MARCINKOWSKI, JIM WEBER and  ROBERT CAUSEY. Finally, the Matthew Highlands Food Science Pilot Plant has become more economically viable and of service to Maine's food industries since the arrival or Robert Dumas, the new Pilot Plant manager \textbf{(SVV 2.2.1, 2.2.2, \& 2.2.3)}.\\~\\ 
\item[\textbf{2.3}] \textbf{We will communicate effectively with all stakeholders.}\\~\\
	STEPHANI BURNETT memberships and engagement
	LILY CALDERWOOD UMaine wild blue berry conference online, daily reports
	MARY ELLEN CAMIRE Portland press herald, Mainewomenmagazine, AHA news
TIM UNDERGRADUATE COORDINATOR AND GRAD COORDINATOR. SERVICE to aquaculture
MARY ELLEN CAMIRE extensive editor responsibilities 
ERIC GALLANDT associate editor of organic farming 
CHARLENE GRAY  Undergraduate coordinator in SAG and ENH programs. Successful recruitment initiatives
PAULINE KAMATH - teen science cafe
	DOROTHY KLIMAS-ZACAS Affiliatins with many countries, US, UK, Greece, Israel.
ANNE LICHTENWALNER - Promoting sustainability by minimizing use of antimicrobials
JADE MCNAMARA Interview with Fox 22 Bangor on current funded research grant 
EILEEN MOLLOY Maine Academy of Nutrition and Dietetics
https://www.eatrightmaine.org/conference
National Food and Nutrition Expo and Conference
https://community.eatrightpro.org/events/event-
May 7, 2021, WGME TV news crew (Dustin Bonk and cameraman) filmed a story at Highmoor Farm on peach breeding for cold hardiness.
%SUE SULLIVAN has effectively organized the undergraduate coordinators to reach out to prospective students, while at the same time we have forwarded positive research news to Erin Miller in the Dean's office \textbf{(SVV 2.3.1)}. We have also updated the SFA webpage and worked with admissions to prepare effective video presentations and virtual tours of our facilities on the UMaine website \textbf{(SVV 2.3.2)}. As part of our ``Define Tomorrow" proposal we have proposed increased coordination with other campuses in the UMaine system \textbf{(SVV 2.3.3)}.   
\end{description}
\item[\textbf{3.}] \textbf{The university will be a rewarding place to live, learn and work by sustaining an environment that is diverse and inclusive, and fosters the personal development of all its stakeholders.}
\begin{description}[topsep=11pt, noitemsep]
\item[\textbf{3.1}] \textbf{We will be recognized as a great place to work in Maine.}\\~\\
	CHARLENE GRAY Expanding your horizons workshop for girls, Divesrity. Sustainability - UMCE zoom confenerences, Maine Question Podcast
	SUE ISHAQ - Pod Coordinator for the Orono chapter of 500 Women Scientists
	DOROTHY KLIMIS-ZACAS Fulbright Scholarship program.
	ANNE LICHTENWALNER - Member of WiSTEMM
	JENNIFER PERRY - Acitive in WiSTEMM and in DEI training
	BRYAN PETERSON - Instructor, Master Gardener Core Competencies Lesson, UMaine Campus. September 2020. I was invited to deliver a one-day course, Plant Propagation, as part of a training series for Master Gardeners.
	RACHEL SCHATTMANLed a group of faculty/graduate student collaborators to assess needs and opportunities for a new, transdisciplinary graduate fellows’ program to be housed at the George J. Mitchell Center. Assessment included surveys of graduate students, faculty, and collaborators outside of the university, and focus groups with Mitchell Center community members and graduate students (2021) 
	DENISE SKONBERG - President's Council for Diversity Equity and Inclusion
%As part of the NSFA Roadmap, the School of Food and Agriculture held a half day retreat in January 2020 to discuss Diversity, Equity and Inclusion. In addition several of our faculty are Safe Zone trained. As a way to build community we have social gatherings for all our faculty and staff. Major success this past year were the recruitment of ROBERT DUMAS, RACHEL SCHATTMAN, AND SUE ISHAQ, all from out of state. In addition we have three pre-tenured faculty who, as a cohort, are positioned well for an early tenure application (JENNIFER PERRY, PAULINE KAMATH, AND JUAN ROMERO). This speaks to the nurturing environment we provide young faculty \textbf{(SVV 3.1.1, 3.1.2, \&  3.1.3)}.\\~\\ 
	
\item[\textbf{3.2}] \textbf{Students will form a lifelong relationship with the university.}\\~\\
%Our unit has many student clubs which provide extracurricular activities within the shelter of the university. Many activities take place at our farms and green houses with plants, animals and food representing a shared experience for faculty, staff, and students. These activities are typical of the Maine way of life and serve to link the student's past and future experiences through their present at the University \textbf{(SVV 3.2.1 \& 3.2.2)}. 
\end{description}
\end{description}~\\
\textbf{Summary}\\~\\
%In spite of gains made last year, AY 2020-2021  will be challenging due to the aftermath of the Covid-19 pandemic. Led by the SAG faculty, we are transitioning to Brightspace for the upcoming year, and will be prepared to teach classes online as required, while remaining flexible to manage face-to-face instruction. The diversity of our unit is one of its strengths, and we hope to build on that strength as we move forward with campus wide initiatives to embrace diversity, equity and inclusion. We will continue to explore ways to sustain and increase enrollment. One possibility under consideration is to reach out to urban youth, for whom a degree from SFA may be a key step in upward mobility. With several retirements on the horizon we must maintain our focus on recruitment of new faculty, with the pressing need for a new research faculty member in FSN, and an instructor in AVS.  \\~\\ 
\textbf{Goals for the next year}
\begin{description}[topsep=11pt, noitemsep]
\item[\textbf{1:}] Green house
\item[\textbf{2:}] %Approval of new AVS instructor and search
\item[\textbf{3:}] Sheep club
\item[\textbf{4:}] %Increase enrollment through targeted outreach
\end{description}
\newpage
\textbf{Appendix I}\\~\\
\begin{turn}{0}
\includegraphics[scale=0.7]{HERB_2019_Approval_letter-UMaine.pdf}
\end{turn}
\newpage
\textbf{Appendix II}\\~\\
\begin{turn}{0}
\includegraphics[scale=0.4]{Dietetic_Intern_passing_rates.jpg}
\end{turn}
\newpage
\textbf{Appendix III}\\~\\
\begin{turn}{0}
\includegraphics[scale=0.75]{Summary_of_Program_Reviews_2019-2020_SFA.pdf}
\end{turn}
\end{document}
