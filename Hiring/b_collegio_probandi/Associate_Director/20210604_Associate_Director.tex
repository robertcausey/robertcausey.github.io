\documentclass[11pt]{article}
\usepackage{geometry}
\usepackage[comma,sort&compress]{natbib}
\usepackage{marvosym}
\usepackage{rotating}
\usepackage{graphics}
\usepackage{amsmath}
\usepackage{epsfig}
\usepackage{enumitem}
\usepackage{textcomp} %needed for /texttildelow command
%\usepackage{picins,graphicx}
\usepackage{txfonts}
\usepackage[english]{babel}
\geometry{letterpaper, margin=0.75 in}
\renewcommand{\rmdefault}{ptm} % Arial
\renewcommand{\sfdefault}{phv} % Arial
\def\changemargin#1#2{\list{}{\rightmargin#2\leftmargin#1}\item[]}
\let\endchangemargin=\endlist 
\begin{document}
\sloppy \rmfamily
\begin{center}\textbf{NSFA Faculty Position Request\\Date of Last Edit:~}June 4, 2021\end{center}

\noindent\textbf{Unit:~}SFA\hfill\textbf{Title:~}Associate Director School of Food and Agriculture\hfill \textbf{Start Date:~}Aug 1, 2022\\~\\
\textbf{How does this position address needs in academic programming (i.e., courses taught)?}\\~\\
\small\sffamily 
This request is to replace a vacancy left by the promotion of our current Associate Director, Susan Sullivan, to the role of Associate Dean. Responsibilities for the position are as follows:   
\\~\\
The Associate Director reports to the Director of the School of Food and Agriculture (SFA) and is primarily responsible for administration of academic programs in conjunction with the program coordinators and academic staff.  Desired qualifications for this position include a degree in a relevant area (Ph.D. desired, M.S. required) and significant academic administrative experience.  The Associate Director is expected to have experience related to the broad area of food and agriculture and also to have academic experience, i.e., teaching and advising in a University setting.  We also expect the Associate Director to have excellent communication skills and to be able to build effective collaborative working relationships with students, staff, faculty members, administrators, and others.  Duties include:\\~\\
1. University of Maine Liaison.  The Associate Director attends meetings of the Executive Committee of the College, meetings of academic Program Coordinators, and serves on the College Curriculum Committee.  The Associate Director serves on other ad hoc or continuing committees as appropriate, and meets with others in administration across the campus as needs arise.  The Associate Director initiates and responds to requests for information flow between SFA and other offices.
\\~\\
2. Personnel.  The Associate Director assists the Director in supervision and evaluation of SFA support staff.
\\~\\
3. Administrative Functions.  The Associate Director is responsible for administrative tasks including: hiring and supervision of hourly-wage graders to support teaching, approving time for SFA staff, leading and serving on search committees, and taking minutes at faculty and program curriculum committee meetings.  Other administrative functions, such as fulfilling the duties of building manager or safety officer, may be included at the discretion of the School Director.
\\~\\
4. Undergraduate Programs.  The Associate Director has oversight responsibility for the School’s undergraduate teaching programs, working with Program Coordinators and other staff.  In particular, the Associate Director offers leadership and coordination in the following areas:
\begin{quote}
	~\\
A. Recruitment and Retention. The Associate Director develops recruitment strategies and materials, coordinates responses to School-level inquiries, organizes mailings and telephone calling, and arranges faculty participation in open houses, orientation, and other academic events.  The Associate Director may engage in recruiting off campus from time to time and may advise student clubs.  The Associate Director manages a Student Ambassador program to support recruiting efforts.
\\~\\
B. Advising and Teaching. The Associate Director oversees the assignment of permanent faculty advisors to undergraduate students and ensures the effectiveness of the School’s undergraduate advising.  The Associate Director may serve as the academic advisor for some undergraduate students.  The Associate Director mentors new faculty in teaching and advising and adjunct faculty in teaching.  The Associate Director receives feedback from students and resolves complaints.
\\~\\
C. Curriculum Development. The Associate Director chairs the SFA Undergraduate Academic Affairs Committee and may assist with curriculum development.  Curriculum decisions remain with Program Coordinators and degree program faculty.  The Associate Director has signature approval for new courses and course modification proposals and is responsible for ensuring appropriate assessment and documentation of program goals and learning outcomes.
\\~\\
D. Course Scheduling. The Associate Director oversees the collection of information from faculty with which to build a schedule of classes and the entry of that data into Infosilem scheduling software.  The Associate Director is responsible for updating course constraints and course combinations, in consultation with faculty, and coordinates the entire schedule of classes to meet program and individual faculty needs.
\\~\\
E. Supporting Undergraduate Program Coordinators. The Associate Director trains, supports and covers for program coordinators as needed and assists with initiatives for each program.
\\~\\
F. Support for Experiential Learning. The Associate Director manages applications and awards for the Barrett Work Merit Scholarship, the Alfred A. Bushway Undergraduate Research and Testing Fund, and the J. Franklin Witter Undergraduate Research Endowment Fund.\\
\end{quote}
5. Teaching.  The Associate Director contributes to the School’s teaching mission, offering one course per semester based on their expertise. 
\\~\\
\textbf{\rmfamily How does this position support the research mission of the unit?}\\~\\
The Associate Director does not have a research appointment. However, they may be involved in graduate education as a member of the graduate faculty.
\\~\\
\textbf{\rmfamily In what ways does this position support College, University, or System initiatives and/or priorities?}
\\~\\
This position directly supports initiatives of the College and Campus in promoting excellence in undergraduate advising and program quality; recruitment and retention; diversity, equity and inclusion; and experiential learning. \\
\\~\\
This position directly supports all three of the UMaine System R \& D Goals for FY 20-24:\\~\\
\#1.~Make Maine the best state in the nation to live, work, and learn by 2030. \\
\#2.~Establish an innovation-driven Maine economy for the 21st century.\\
\#3.~Prepare the knowledge-and-innovation workforce for Maine.\\~\\
\vfill
\noindent\textbf{\rmfamily Can this position meet any partner accommodation needs (describe)?}
\\~\\Not that we are aware of currently.\\~\\
\vfill
\noindent\textbf{\rmfamily In what ways does this position support interests of the State, including workforce development?}\\~\\
This position supports the interests of the State through leadership of our educational programs academic programs in the areas of Food Science and Human Nutrition, Sustainable Agriculture, Environmental Horticulture, and Animal and Veterinary Sciences.
\\~\\
\vfill
\newpage\noindent\textbf{\rmfamily What are the anticipated startup needs (items, not just total dollar value)?}\\~\\
None
\vfill
\noindent\textbf{\rmfamily Considering all available sources, how do you anticipate meeting the startup needs?}\\~\\
Not applicable\\~\\
\vfill
\noindent\textbf{\rmfamily What are the anticipated space needs?}\\~\\ Office only\\~\\
\vfill
\noindent\textbf{\rmfamily How do you anticipate meeting the space needs?} \\~\\
Office space vacated by Sue Sullivan would be available.\\~\\
\vfill
\noindent\textbf{\rmfamily Was this position request discussed and approved by the unit faculty?} \hfill ~ \hfill \textbf{\Large{\CrossedBox} \normalsize{Yes}} \hfill \textbf{\Large{\HollowBox} \normalsize{ No}}\\~\\%table373\CrossedBox_tocheckbox
\end{document}
