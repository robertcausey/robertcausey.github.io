\documentclass[11pt]{article}
\usepackage{geometry}
\usepackage[comma,sort&compress]{natbib}
\usepackage{marvosym}
\usepackage{rotating}
\usepackage{graphics}
\usepackage{amsmath}
\usepackage{epsfig}
\usepackage{enumitem}
\usepackage{textcomp} %needed for /texttildelow command
%\usepackage{picins,graphicx}
\usepackage{txfonts}
\usepackage[english]{babel}
\geometry{letterpaper, margin=0.75 in}
\renewcommand{\rmdefault}{ptm} % Arial
\renewcommand{\sfdefault}{phv} % Arial
\def\changemargin#1#2{\list{}{\rightmargin#2\leftmargin#1}\item[]}
\let\endchangemargin=\endlist 
\begin{document}
\sloppy \rmfamily
\begin{center}\textbf{NSFA Faculty Position Request\\Date of Last Edit:~}October 24, 2020\end{center}

\noindent\textbf{Unit:~}SFA\hfill\textbf{Title:~}Assistant Professor in Food Engineering\hfill \textbf{Start Date:~}Sep 1, 2021\\~\\
\textbf{How does this position address needs in academic programming (i.e., courses taught)?}\\~\\
\small\sffamily 
\\~\\
This position request is to replace a vacancy left by the departure of one of our faculty (Balunkeswar Nayak).  The position teaches course-work to meet certification of our curriculum by the Institute of Food Technologists for its Approved Undergraduate Programs. Not filling the position will make it difficult to maintain IFT accreditation, which draws out-of-state students. Currently the State of Maine does not have the expertise, in either full-time or adjunct faculty, to deliver classes in food engineering.
\\~\\
The primary teaching responsibilities of this position include: FSN 485 Introduction to Food Engineering Principles; FSN 486 Food Engineering Lab; FSN 502 Food Preservation; and a graduate level class in Food Science and/or Engineering to meet curricular needs, including the UMaine Gold Program. The teaching load will average 9 credits a year. Graduate supervision is an important part of this position. Responsibilities will include research, teaching, student advising, and providing expertise to the state and UMaine community in the areas of Food Science and Food Engineering.
\\~\\
\textbf{\rmfamily How does this position support the research mission of the unit?}\\~\\
The position will conduct research to enhance the Maine Food and Bioproducts Industry, in areas such as novel food processing techniques, food applications of cellulose nanofibrils, nano-delivery of food bioactives, and commercialization of emerging technologies. Applicants are expected to build a robust, externally-funded research program and create collaborations within the School, Cooperative Extension, Advanced Manufacturing Center, Graduate School of Biomedical Science and Engineering, farmers, food and beverage manufacturers, and with other organizations such as Maine Manufacturing Extension Partnership (MMEP), The Maine Organic Farmers and Gardeners Association, Maine Potato Board, Wild Blueberry Commission of Maine, New England Food Processors Community of Practice, Alliance for Maine's Marine Economy, and Maine Seaweed Alliance. The position would help to develop solutions for establishing processing infrastructure that is lacking in the state. This position will provide mentorship to Doctoral and Master’s students, and provide research opportunities for undergraduates.
\newpage
\noindent\textbf{\rmfamily In what ways does this position support College, University, or System initiatives and/or priorities?}
\\~\\
This position directly supports three Initiatives of the College Roadmap to Excellence:\\~\\
\#1.~Excellence in undergraduate advising and program quality by hiring faculty to meet IFT certification.\\
\#2.~Growing our research enterprise in the service of science and Maine by building a robust, externally-funded research program in food engineering.\\
\#3.~Enhancing quality and impact of our graduate education programs through scientific discovery and innovation in the field of food science while benefiting Maine food producers.
\\~\\
The position supports the University's strategic values and vision statements by supporting learners in experiential learning and innovation in food and nutrition (Goals 1.1 and 1.2), supports key infrastructure such as the Matthew Highlands Pilot plant for food innovation (Goal 2); and through using the hire to support diversity, equity and inclusion (Goal 3).
\\~\\
This position directly supports all three of the UMaine System R \& D Goals for FY 20-24:\\~\\
\#1.~Make Maine the best state in the nation to live, work, and learn by 2030, by improving food processing for Maine consumers and Maine food producers. \\
\#2.~Establish an innovation-driven Maine economy for the 21st century by advances in food processing to support the Maine food industry.\\
\#3.~Prepare the knowledge-and-innovation workforce for Maine by training qualified graduates and undergraduates to enter the Maine food and bioprocessing industry workforce.\\~\\
\vfill
\noindent\textbf{\rmfamily Can this position meet any partner accommodation needs (describe)?}
\\~\\Not that we are aware of currently.\\~\\
\vfill
\noindent\textbf{\rmfamily In what ways does this position support interests of the State, including workforce development?}\\~\\
This position supports the food security of Maine's population, Maine's food producers, and educates students to enter Maine's diverse and expanding food and beverage industry.
\\~\\
\vfill
\newpage\noindent\textbf{\rmfamily What are the anticipated startup needs (items, not just total dollar value)?}\\~\\
Laboratory equipment for nutritional research - \$150K \\
Graduate student support - one graduate assistant for three years \$60K\\
Summer salary - \$25K\\~\\
\vfill
\noindent\textbf{\rmfamily Considering all available sources, how do you anticipate meeting the startup needs?}\\~\\
Support from the School, College, and Vice President for Research\\~\\
\vfill
\noindent\textbf{\rmfamily What are the anticipated space needs?}\\~\\ Laboratory, office and graduate student space\\~\\
\vfill
\noindent\textbf{\rmfamily How do you anticipate meeting the space needs?} \\~\\
Laboratory and office space vacated by Dr. Nayak would be available. Graduate students would be housed in the FSN graduate space in Hitchner 100.\\~\\
\vfill
\noindent\textbf{\rmfamily Was this position request discussed and approved by the unit faculty?} \hfill ~ \hfill \textbf{\Large{\CrossedBox} \normalsize{Yes}} \hfill \textbf{\Large{\HollowBox} \normalsize{ No}}\\~\\%table373\CrossedBox_tocheckbox
\end{document}
sition will conduct research in Nutritional Biochemistry/Nutrigenomics, focused on translational research to improve human health, especially in Maine's population. Applicants are expected to build a robust, externally-funded research program and create collaborations within the School, across campus, and with other institutions. This position has the potential to have a positive impact on the health of Maine's population, and on Maine's food producers through many ways, including identifying  nutritional benefits of Maine produce.
