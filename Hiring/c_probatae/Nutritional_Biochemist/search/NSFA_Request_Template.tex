\documentclass[11pt]{article}
\usepackage{geometry}
\usepackage[comma,sort&compress]{natbib}
\usepackage{marvosym}
\usepackage{rotating}
\usepackage{graphics}
\usepackage{amsmath}
\usepackage{epsfig}
\usepackage{enumitem}
\usepackage{textcomp} %needed for /texttildelow command
%\usepackage{picins,graphicx}
\usepackage{txfonts}
\usepackage[english]{babel}
\geometry{letterpaper, margin=0.75 in}
\renewcommand{\rmdefault}{ptm} % Arial
\renewcommand{\sfdefault}{phv} % Arial
\def\changemargin#1#2{\list{}{\rightmargin#2\leftmargin#1}\item[]}
\let\endchangemargin=\endlist 
\begin{document}
\sloppy \rmfamily
\begin{center}\textbf{NSFA Faculty Position Request\\Date of Submission:~}Sept 30, 2020\end{center}

\noindent\textbf{Unit:~}SFA\hfill\textbf{Title:~}Lecturer in Animal and Veterinary Sciences \hfill \textbf{Start Date:~}Jan 1, 2021\\~\\
\textbf{How does this position address needs in academic programming (i.e., courses taught)?}\\~\\
\small\sffamily The foundational animal science class at UMaine is AVS 145 Introduction to Animal Science (lecture), with AVS 146 as the corresponding laboratory class.  Due to Covid-19 the lecture component (AVS 145) will be delivered remotely. Incoming first year students need opportunities for learning that are face to face due to the high number of remote classes. We therefore plan to use AVS 146 to create a sense of community among the AVS students, and to provide an engaging, hands-on experience with animals that will help them during an unusual first semester at UMaine.  Students will be provided some hands on exercises with animals during scheduled labs (4 lab sections, 20 students per lab). In addition, the whole class and will be rotated through the farm, one or two students at a time, to observe farm operations.\\~\\
\textbf{\rmfamily How does this position support the research mission of the unit?}\\~\\
\newpage
\noindent\textbf{\rmfamily In what ways does this position support College, University, or System initiatives and/or priorities?}
\\~\\
\textbf{\rmfamily Can this position meet any partner accommodation needs (describe)?}\\~\\
\textbf{\rmfamily In what ways does this position support interests of the State, including workforce development?}\\~\\
\newpage\noindent\textbf{\rmfamily What are the anticipated startup needs (items, not just total dollar value)?}\\~\\
\textbf{\rmfamily Considering all available sources, how do you anticipate meeting the startup needs?}\\~\\
\textbf{\rmfamily What are the anticipated space needs?}\\~\\
\textbf{\rmfamily How do you anticipate meeting the space needs?} \\~\\
\textbf{\rmfamily Was this position request discussed and approved by the unit faculty?} \hfill ~ \hfill \textbf{\Large{\CrossedBox} \normalsize{Yes}} \hfill \textbf{\Large{\HollowBox} \normalsize{ No}}\\~\\%table373
\end{document}
\textbf{Covid-19 Hazard Assessment}\\~\\
The environment at the Witter Center may be safer, from a Covid-19 perspective, than gyms, libraries, classrooms, and other sites on campus. First, students at farms and field sites are under supervision, have accepted the culture of safety training, and will comply with posted protocols. Second, the open air environment of these spaces makes social distancing possible and minimizes aerosolized spread of the virus. Third, only a few students (e.g. 5 or less) are typically on-site at any time, often in different parts of the farm. The risk of transmission is therefore minimized.\\~\\
The greatest risk of Coronavirus transmission at Witter arises from student contact in classes, clubs, or study groups which share the Witter classroom, conference room, bathrooms, and office area. This creates increased risk for farm staff, which then jeopardizes welfare of the animals. \\~\\The risk of transmission in the barns during animal care duties, and labs is significantly lower than in the classroom, but nevertheless is a cause for concern. Transmission could occur in the barns through handling shared equipment or when individuals come into close proximity working around an animal.\\~\\    
\textbf{Hierarchy of Controls}\\~\\
Steps that can be taken to minimize these risks are outlined below in the context of a Hierarchy of Controls. If these practices are followed, it is likely that students at the Witter Center will be as safe as students on campus, or possibly safer.\\ 
\begin{description}[topsep=11pt, noitemsep]
\item[\textbf{1.}] \textbf{Elimination }\\~\\
Students in AVS 146 will not have access to the office areas or classrooms of the Witter Center, and will have no reason to go to these areas, except in unusual circumstances.\\ 
\item[\textbf{2.}] \textbf{Substitution}\\~\\Scheduled labs and class communications will be managed through the Brightspace Portal, connected to Google Drive, Zoom, and UMaine's Kaltura video platform. Scheduled labs will all be held at the Chute Center, outdoor arena, or in the lanes around them. Each section subdivided into 4 groups of 5 students, each group being a minimum of 30 feet distance from each other. \\ 
\item[\textbf{3.}] \textbf{Engineering Controls.}\\~\\To protect farm staff, and meet student needs, toilet facilities will be provided using two portable toilets located outside the horse barn. The cost will be covered using the SFA and equine teaching budget. Students would be encouraged to plan their activities to minimize use of these facilities.\\~\\Temporary storage space for tack and equine supplies kept in the tack room will be established in the horsebarn. This will eliminate the need for students to go into the office area to collect equine tack and equipment.\\~\\Lab manuals will be online and accessible via mobile phones, tablets, etc through the Brightspace portal. \\ 
\item[\textbf{4.}] \textbf{Administrative Controls.}\\~\\
Care for the horses is scheduled in the morning, noon, and evening. No more than 5 students will be present at any one time. These activities take approximately two hours or less, after which students leave the farm.  \\~\\Students will be required to provide their own transportation to the farm. However, students who are roommates or family members may share transportation. Students without vehicles will have to walk to the farm (\texttildelow 30 minutes). Walking groups will be set up so no student walks to the farm alone. Labs activities will be scheduled to accommodate time spent in walking to and from farm.   \\~\\
Laboratory sections in AVS 146 will consist of 5 students per lab station, with 1 animal at each station either tied or held on a long lead (greater than 6 feet in length). This will allow students to interact with the animal, and pass the animal to one another, without coming onto contact with a handler or another student. Disinfectant wipes will be available to wash hands or gloves after handling. Animal handling may take take place in the Chute Center (indoor riding arena), the outdoor arena, or in lanes outside, providing options for dispersing the group over the maximum distance. \\~\\ 
The course instructor will be accountable to the Livestock Operations Supervisor to whom assurances will be provided that farm policies are being followed.\\ 
\item[\textbf{5.}] \textbf{Personal Protective Equipment (PPE).}\\~\\Masks and social distancing will be mandatory.\\
\end{description}
\noindent\textbf{Summary}\\~\\
Above is a protocol for reintroducing safely students to the Witter Center in the experiential courses AVS 146. Adhering to the proposed Hierarchy of Controls should reduce the likelihood of transmission of Covid-19 to a level equal to, or less than, other campus sites. 
\end{document}	  
