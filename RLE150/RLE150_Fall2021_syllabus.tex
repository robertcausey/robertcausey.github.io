\documentclass[11pt]{article}
\usepackage{geometry}
\usepackage{hyperref}
\usepackage[comma,sort&compress]{natbib}
\usepackage{rotating}
\usepackage{graphics}
\usepackage{amsmath}
\usepackage{epsfig}
\usepackage{enumitem}
%\usepackage{picins,graphicx}
\usepackage{txfonts}
\usepackage[english]{babel}
\geometry{letterpaper, margin=0.8 in}
\renewcommand{\rmdefault}{ptm} % Arial
\renewcommand{\sfdefault}{phv} % Arial
\def\changemargin#1#2{\list{}{\rightmargin#2\leftmargin#1}\item[]}
\let\endchangemargin=\endlist 
\begin{document}
\sloppy \rmfamily
\begin{center}\textbf{SYLLABUS}\end{center}
\textbf{Course Title:}\\
RLE 150  Research Learning Experience  - 1 cr Greener Pastures: Practice, policy and perception in animal agriculture\\~\\
\textbf{Catalog Description:}\\
	
\noindent EXPERIMENTAL FALL 2021
Research Learning Experiences (RLE) offer first-year and second-year students from all majors the opportunity to have meaningful involvement in research, to grow intellectually, and to develop a foundational understanding of methods of inquiry and engagement across the disciplines. A wide variety of RLE offerings introduce students to research and creative practices in the natural and social sciences, the humanities, engineering, and the arts. This one-credit course will encourage students, sense of belonging and engagement, and jumpstart their academic careers.\\~\\
\textbf{\emph{Description for this section -}} This class is intended to stimulate curiosity of students in any academic discipline,  in animal agriculture. Research areas include 1. Questions related to improving agricultural production, animal welfare, and minimizing environmental impact; 2. Legislative, regulatory and ethical considerations of animal agriculture; 3. What is the human experience of animal agriculture, and how might his be expressed, explored, or engaged through the arts or personal reflection.\\~\\
\textbf{Semester:}\\Fall 2021\\~\\
\textbf{Background to this Section:}\\  Animal agriculture has been part of human experience since the beginning of history. Cave paintings, the archeological record, and oral traditions reveal our complex and evolving relationship with animals. This relationship creates a diverse array of challenges: how to feed and clothe ourselves, make a profit, sustain animal resources, promote animal welfare, and preserve the environment. But to these practical questions can be added experiential ones: how do we feel about animals in the way our lives touch them? Do we have a sense of guilt, pleasure, awe, fear etc in our treatment of animals? Exploration of these feelings might help us to improve our experience with animals, and their experience of us.\\~\\
The purpose of this class is to stimulate curiosity of participants from any academic discipline, (physical sciences, biological sciences, social sciences, arts etc), in the challenges posed by animal agriculture. Students will formulate a research question specific to animal agriculture, according to the disciplines of science, social science, or the arts; apply research techniques relevant to the discipline; and communicate findings in the format commonly used by the discipline. \\~\\
Research areas include 1. Practice: Questions related to improving agricultural production, animal welfare, and minimizing environmental impact; 2. Policy: Legislative, regulatory and ethical considerations of animal agriculture; 3. Perception: What is the human experience of animal agriculture, and how might his be expressed, explored, or engaged through the arts or personal reflection.\\~\\
\newpage
\noindent \textbf{INSTRUCTIONAL PERSONNEL}\\~\\
\textbf{Course Instructor}:\\
Robert Causey\\
Phone: 207-922-7475,  Email: rcausey@maine.edu\\~\\
\textbf{Graduate Assistant - poultry and cattle}\\
Maddy Philbrick \\
Phone: 207-745-2762  email: madison.philbrick@maine.edu\\~\\
\textbf{Graduate Assistant - perception projects}:\\
Katherine Dubois:\\
Phone: 207-551-6496 email: duboiska@miamioh.edu\\~\\
\textbf{Undergraduate Assistant - horses}:\\
Cleo Horvath\\
Phone: 201-686-6517  email: cleo.horvath@maine.edu\\~\\
\textbf{Undergraduate Assistant - pigs}:\\
Morgan Belvin\\
Phone: 201-803-1586  email: morgan.belvin@maine.edu\\~\\
\noindent\textbf{Classroom Civility:}\\ Any successful learning experience requires mutual respect on behalf of the student and the instructor. The instuctor, as well as the fellow students, should not be subjected to any student's behavior that is in any way disruptive, rude, or challenging to the instructor's authority in the classroom. A student should not feel intimidated or demeaned by his/her instuctor and students must remember that the instuctor has primary responsibility for control over classroom behavior and maintenance of academic integrity. The instuctor can order the temporary removal or exclusion from the classroom of any student engaged in disruptive conduct or conduct violating the general rules and regulations of the institution.\\~\\
\textbf{Course Schedule:}\\
The class will meet weekly at 4 pm every Tuesday. Other activities related to research will be arranged.\\~\\
\noindent\textbf{Course Goals:}\\
This course is intended to give you hands - on experience in practical in research related to Agriculture.\\~\\
\noindent\textbf{Learning Outcomes:}\\After attending this class, students will be able to:\\~\\
\textbf{\emph{Cognitive}}
\begin{description}[topsep=11pt, noitemsep]
\item[\textbf{1.}] Students will explain the importance of belonging to a community for their learning. Students will formulate appropriately scoped topics or questions that will guide their scholarly exploration. 
\item[\textbf{2.}] Students will describe their iterative approaches to exploration for work without a defined answer.
\item[\textbf{3.}] Students will effectively communicate with collaborators about their experience. 
\item[\textbf{4.}] Students will demonstrate responsibility for the discovery process. 
\end{description} 
\newpage \noindent
\textbf{\emph{Dispositional}}
\begin{description}[topsep=11pt, noitemsep]
\item[\textbf{1.}] Students will identify the relevance or applicability of their experience beyond the RLE course. 
\item[\textbf{2.}] Students will reflect on how their exploratory process has helped them develop as learners. 
\item[\textbf{3.}] Students will explain the importance of belonging to a community for their learning.
\end{description}
\noindent \textbf{Course Materials:}\\ 
All course materials will be made available through robertcausey.github.io. \\~\\
\textbf{Clothing:}\\ 
Steel-toe work boots are recommended if you are an AVS major. For this class sturdy leather shoes or boots are best. Working cowboy boots are ideal. If you are working with animals, especially cattle, pigs, sheep or poultry, long sleeve clothing and pants will be required. When working with horses on the ground, shorts and short sleeves are acceptable if the weather is hot, but in general, long sleeves and long pants are preferred for protection against ticks and abrasions. Leather work gloves are highly recommended. When the weather gets cold be prepared to dress warmly.\\~\\ 
\textbf{Directions to the farm:}\\
Here is the street address, and a link to the google maps location of the farm, if you are driving.\\~\\
160 University Farm Road, Old Town, Maine - https://goo.gl/maps/iuXuwRKcxnnGCHiJ7\\~\\
\textbf{Transportation to Farm:}\\
Students will have to provide their own transportation to the farm. Because of Covid we cannot offer transportation in a University owned vehicle. It takes 30 minutes to walk to the farm on a well traveled path through the  University forest. \\~\\ 
\textbf{Area Specific Safety Training:}\\ Students are required to attended area specific safety training at the Witter center in person if they are taking the class on campus. Students that fail to complete this training will not be able to participate in farm activities. Please contact the course instructor if you have not taken this training and plan to attend this class on campus at the Witter Center. \\~\\
\textbf{Annual Basic and Farm Safety Training Certificates:}\\ All students are required to take the Annual Basic Safety Training by September 15th. Go to http://sem.umaine.edu/safety-training/ and follow instructions. Similarly, all students are required to take the Farm Safety Training by September 15th. Go to http://sem.umaine.edu/safety-training/and follow instructions [Farm Safety (Agricultural Training Requirements)]. Submit both certifications as pdf or jpg files to Brightspace AVS 146. Students that do not comply will not be able to participate in farm activities and will automatically get zero in their reports and attendance, resulting in failure of the class.\\~\\ 
As a final project, which will count as your final exam, you will map out a potential career for yourself in Animal Science and \textbackslash or Veterinary Medicine. The key component will be to outline a feasible career option that will support you economically. \\~\\ 
\textbf{Warnings specific to working on farms:}\\ Students who have concerns about their allergy status to animals are encouraged to check with their physician. Also, immunocompromised students are encouraged to consult with their physician about potential zoonotic disease transmission.\\~\\ 
\noindent
\textbf{Academic Honesty Statement:} \\Academic honesty is very important. It is dishonest to cheat on exams, to copy term papers, to submit papers written by another person, to fake experimental results, or to copy or reword parts of books or articles into your own papers without appropriately citing the source. Students committing or aiding in any of these violations may be given failing grades for an assignment or for an entire course, at the discretion of the instructor. In addition to any academic action taken by an instructor, these violations are also subject to action under the University of Maine Student Conduct Code.  The maximum possible sanction under the student conduct code is dismissal from the University.Please see the University of Maine System’s Academic Integrity Policy  listed in the Board Policy Manual as Policy 314: https://www.maine.edu/board-of-trustees/policy-manual/section-314/ \\~\\ 
\textbf{Students Accessibility Services Statement:} \\If you have a disability for which you may be requesting an accommodation, please contact Student Accessibility Services, 121 East Annex, 581.2319, as early as possible in the term. Students who have already been approved for accommodations by SAS and have a current accommodation letter should meet with me (Robert Causey) privately as soon as possible.\\~\\ 
\textbf{Course Schedule Disclaimer (Disruption Clause):}\\ In the event of an extended disruption of normal classroom activities (due to COVID-19 or other long-term disruptions), the format for this course may be modified to enable its completion within its programmed time frame. In that event, you will be provided an addendum to the syllabus that will supersede this version.\\~\\ 
\textbf{Observance of Religious Holidays/Events:} \\The University of Maine recognizes that when students are observing significant religious holidays, some may be unable to attend classes or labs, study, take tests, or work on other assignments. If they provide adequate notice (at least one week and longer if at all possible), these students are allowed to make up course requirements as long as this effort does not create an unreasonable burden upon the instructor, department or University. At the discretion of the instructor, such coursework could be due before or after the examination or assignment. No adverse or prejudicial effects shall result to a student’s grade for the examination, study, or course requirement on the day of religious observance. The student shall not be marked absent from the class due to observing a significant religious holiday. In the case of an internship or clinical, students should refer to the applicable policy in place by the employer or site.\\~\\ 
\textbf{Sexual Discrimination Reporting}\\ 
The University of Maine is committed to making campus a safe place for students. Because of this commitment, if you tell a teacher about an experience of sexual assault, sexual harassment, stalking, relationship abuse (dating violence and domestic violence), sexual misconduct or any form of gender discrimination involving members of the campus, your teacher is required to report this information to the campus Office of Sexual Assault \& Violence Prevention or the Office of Equal Opportunity. If you want to talk in confidence to someone about an experience of sexual discrimination, please contact these resources:\\
\textbf{For confidential resources on campus:} \\Counseling Center: 207-581-1392 or Cutler Health Center: at 207-581-4000.\\ 
\textbf{For confidential resources off campus:\\}  Rape Response Services: 1-800-871-7741 or Partners for Peace: 1-800-863-9909.\\ 
\textbf{Other resources:} \\ The resources listed below can offer support but may have to report the incident to others who can help:\\ 
\textbf{For support services on campus:}\\ Office of Sexual Assault \& Violence Prevention: 207-581-1406, Office of Community Standards: 207-581-1409, University of Maine Police: 207-581-4040 or 911. Or see the OSAVP website for a complete list of services at http://www.umaine.edu/osavp/
\newpage
\begin{center}\textbf{UNIVERSITY OF MAINE COVID-19 SYLLABUS STATEMENT}\end{center}
~\\ 
COVID-19 is an infectious disease caused by the coronavirus SARS-CoV-2. The virus is transmitted person-to-person through respiratory droplets that are expelled when breathing, talking, eating, coughing, or sneezing. Additionally, the virus is stable on surfaces and can be transmitted when someone touches a contaminated surface and transfers the virus to their nose or mouth. When someone becomes infected with COVID-19 they may either have no symptoms or symptoms that range from mild to severe and can even be fatal. During this global pandemic, it is imperative that all students, faculty, and staff abide by the safety protocols and guidelines set forth by the University to ensure the safety of our campus. All students are encouraged to make the Black Bear Cares Pact to protect the health of themselves, the health of others, and the College of Our Hearts Always.\\~\\
Black Bears Care Pact: https://umaine.edu/return/black-bears-care/ \\~\\
\textbf{Symptom checking:} The symptoms of COVID-19 can range from mild to severe, and even people with mild symptoms may transmit the virus to others. Students are encouraged to use the symptom checking app each day before attending class or moving about campus and follow the recommendation prompted within the app. Students should monitor for the following symptoms daily: fever (temperature >100.4F/38.0C) or chills, new cough, loss of taste or smell, shortness of breath/difficult breathing, sore throat, diarrhea, nausea, or vomiting, or the onset of new, otherwise unexplained symptoms such as headache, muscle or body aches, fatigue, or congestion/runny nose.\\~\\
\textbf{Physical distancing:} Students need to make every effort to maintain physical distancing (6 feet or more) indoors and outdoors including within classrooms. The University classrooms and physical spaces have been arranged to maximize physical distancing. Follow the traffic patterns outlined in each building and outdoor space to avoid crowding. If students are in an academic setting (i.e. clinical or lab class) that requires them to reduce physical distancing, they should follow the instructor’s guidelines.\\~\\
\textbf{Face coverings:} Students must wear appropriate face coverings in the classroom. Face coverings must be worn in indoor and outdoor spaces on campus unless people are alone in a room with a door closed or when they are properly physically distanced and do not expect someone to approach them. When face coverings are removed people are placing themselves and those surrounding them at increased risk for COVID-19.\\~\\
\textbf{Physical distancing:} Students need to make every effort to maintain physical distancing (6 feet or more) indoors and outdoors including within classrooms. The University classrooms and physical spaces have been arranged to maximize physical distancing. Follow the traffic patterns outlined in each building and outdoor space to avoid crowding. If students are in an academic setting (i.e. clinical or lab class) that requires them to reduce physical distancing, they should follow the instructor’s guidelines.\\~\\
\textbf{Face coverings:} Students must wear appropriate face coverings in the classroom. Face coverings must be worn in indoor and outdoor spaces on campus unless people are alone in a room with a door closed or when they are properly physically distanced and do not expect someone to approach them. When face coverings are removed people are placing themselves and those surrounding them at increased risk for COVID-19.\\~\\
\textbf{Physical distancing:} Students need to make every effort to maintain physical distancing (6 feet or more) indoors and outdoors including within classrooms. The University classrooms and physical spaces have been arranged to maximize physical distancing. Follow the traffic patterns outlined in each building and outdoor space to avoid crowding. If students are in an academic setting (i.e. clinical or lab class) that requires them to reduce physical distancing, they should follow the instructor’s guidelines.\\~\\
\textbf{Face coverings:} Students must wear appropriate face coverings in the classroom. Face coverings must be worn in indoor and outdoor spaces on campus unless people are alone in a room with a door closed or when they are properly physically distanced and do not expect someone to approach them. When face coverings are removed people are placing themselves and those surrounding them at increased risk for COVID-19.\\~\\
\end{document}
\noindent\textbf{Submitter's Name}\\~\\
Robert Causey\\~\\
\noindent\textbf{Submitter's Email Address}\\~\\
rcausey@maine.edu\\~\\
\noindent\textbf{Submitter's Position}\\~\\
Director - SFA
\\~\\
\noindent\textbf{College or Unit}\\~\\
School of Food and Agriculture
\\~\\
\noindent\textbf{Project/Task}\\~\\
Continuation of Essential Animal Care at J. F. Witter Center (2020-11-25 to 2020-01-25)
\\~\\
\noindent\textbf{Special Requirements}\\~\\
None
\\~\\
\noindent\textbf{Staff}\\~\\
Robert Causey, 207-922-7475, rcausey@maine.edu, Director SFA
Elizabeth McLaughlin, 207-217-1538, elizabeth.mclaughlin1@maine.edu, Livestock Operations Supervisor
Cassie Astle, 207-610-9483, cassieastle73@gmail.com, Equine Trainer
Joshua Hatley, 704-467-2159, joshua.hatley@maine.edu, Farm Superintendent
David Marcinkowski, 207-852-4576, davidmar@maine.edu, Extension Professor
Jim Weber, 207-907-0315, jaweber@maine.edu, Associate Professor
\\~\\
\noindent\textbf{Exact Location of Proposed Activity}\\~\\
\\~\\
\noindent\textbf{Hierarchy of Controls Documents}\\~\\
\\~\\
\end{document}
\noindent\textbf{Background}\\~\\Opportunities for face to face and hands on instruction are at a premium in the era of Covid-19. The farm setting of some classes allows us to offer them as safe and engaging alternatives to remote online learning. 
\\~\\
The horse program at the Witter Center progresses from an elective elementary class (AVS 196), to a required core class (AVS 303), to an elective advanced class (AVS 397). These classes involve online didactic instruction, but include a field experience component in care of the horse herd which is at the J. Franklin Witter Center in Oldtown, ME, a 30 minute walk from the main Orono campus.  AVS 303 involves horse handling, grooming, physical examination, anatomy, locomotion, lameness, and training from the ground. AVS 397 involves training horses from the ground and under saddle. Students in AVS 397 also provide guidance to the less experienced students in AVS 196 and 303 during horse care duties (chores).   \\~\\
\textbf{Covid-19 Hazard Assessment}\\~\\
The environment at the Witter Center may be safer, from a Covid-19 perspective, than many sites on campus. First, students at farms and field sites are under supervision, have accepted the culture of safety training, and will comply with posted protocols. Second, the open air environment of these spaces makes social distancing possible and minimizes aerosolized spread of the virus. Third, only a few students (e.g. 5 or less) are typically on-site at any time, often in different parts of the farm. The risk of transmission is therefore minimized.\\~\\
The greatest risk of Coronavirus transmission at Witter formerly would have arisen from student contact in classes, clubs, or study groups which shared the Witter classroom, conference room, bathrooms, and office area. This created increased risk for farm staff, which then jeopardized welfare of the animals. \\~\\The risk of transmission in the barns during horse care duties, labs, or horse training is significantly less than in classrooms, but nevertheless is a cause for concern. Transmission could occur in the barns through handling shared equipment or when individuals come into close proximity working around a horse.\\~\\    
\textbf{Hierarchy of Controls}\\~\\
Steps that can be taken to minimize these risks are outlined below in the context of a Hierarchy of Controls. If these practices are followed, it is likely that students at the Witter Center will be at signficantly less risk than students on campus.\\ 
\begin{description}[topsep=11pt, noitemsep]
\item[\textbf{1.}] \textbf{Elimination }\\~\\
Students in AVS 196, 303, 397 will not be permitted in the office areas or classrooms of the Witter Center, and will have no reason to go to these areas, except in unusual circumstances.\\ 
\item[\textbf{2.}] \textbf{Substitution}\\~\\Didactic lectures, class meetings and horse care duties will be managed through the Brightspace Portal, connected to Google Drive, Zoom, and UMaine's Kaltura video platform.\\ 
\item[\textbf{3.}] \textbf{Engineering Controls.}\\~\\To protect farm staff, and meet student needs, toilet facilities will be provided using two portable toilets located outside the horse barn. The cost will be covered using the SFA and equine teaching budget. Students would be encouraged to plan their activities to minimize use of these facilities.\\~\\Temporary storage space for tack and equine supplies kept in the tack room will be established in the horsebarn. This will eliminate the need for students to go into the office area to collect equine tack and equipment.\\~\\As part of activities involving lab manuals, forms, or medical records of the horses, these will be online and accessible via mobile phones, tablets, etc through the Brightspace portal.\\ 
\item[\textbf{4.}] \textbf{Administrative Controls.}\\~\\
Care for the horses is scheduled in the morning, noon, and evening. No more than 5 students will be present at any one time. These activities take approximately two hours or less, after which students leave the farm.  Students will be required to provide their own transportation.\\~\\
Laboratory sections in AVS 303 will consist of 3 students per lab, with 1 student per horse. Labs with horses standing will use cross-ties in the aisle-way; labs involving horse handling or training will take place in the Chute Center (indoor riding arena) or outdoor arena. \\~\\ 
The course instructor will be accountable to the Livestock Operations Supervisor to whom assurances will be provided that farm policies are being followed, and that the specific protocols developed by the course instructor for horse care and for student safety are posted and are followed. These protocols include feeding instructions, student safety protocols, maintenance of medical records of the horse herd, IACUC protocols etc.\\~\\
Because offering smaller sized labs presents scheduling challenges, riding opportunities may be reduced. The priority is to care for the horse herd and deliver essential course content for AVS 196, 303, and 397. We will try to offer training under saddle in AVS 397. However, club activities involving riding (Drill Team) will probably not be possible.\\  
\item[\textbf{5.}] \textbf{Personal Protective Equipment (PPE).}\\~\\Masks and social distancing will be mandatory. Helmets are a routine PPE item mandated for working with horses on the ground. To prevent transmission of Covid-19 through shared helmets, each student will be required to have their own helmet.\\
\end{description}
\noindent\textbf{Summary}\\~\\
Adhering to the proposed Hierarchy of Controls listed above should reduce the likelihood of transmission of Covid-19 to a level substantially less than on  campus. 
\end{document}	  
